\documentclass[10pt,a4paper,draft,oneside,notitlepage]{article}
\usepackage[utf8]{inputenc} % Acentos, etc
\usepackage[spanish]{babel} % Idioma

\usepackage[left=2.5cm,right=2.5cm,top=2.5cm,bottom=2.5cm]{geometry} % Margenes
\usepackage{mathptmx} % Times New Roman

\usepackage{multicol} % Dos columnas
\setlength\columnsep{1cm}

\usepackage{titlesec} % Formato de titulos de secciones
\titleformat{\section}{\bfseries\large}{\thesection.}{0.3em}{}
\titleformat{\subsection}{\bfseries\normalsize}{\thesubsection.}{0.3em}{}

\pagestyle{empty} % Sin número de página

\makeatletter
% Metadatos
\def\email#1{\gdef\@email{#1}}
\def\@email{\@latex@warning@no@line{No \noexpand\email given}}

\def\universidad#1{\gdef\@universidad{#1}}
\def\@universidad{\@latex@warning@no@line{No \noexpand\universidad given}}

% Título
\newcommand{\titulo} {
	\begin{center}
		{\Large\bfseries\@title}
		
		\vspace{2em}
		{\large
			\@author

			\itshape
			\@universidad

			\@email
		}
		\vspace{2em}
	\end{center}
}
\makeatother

% Minisección (para las palabras clave, agradecimientos, etc)
\newcommand{\miniseccion}[1] {
	\vspace{1em}
	\noindent\textbf{#1}

	\noindent\ignorespaces
}

% Abstract
\newcommand{\resumen}[1] {
	\begin{center}
	\bfseries\large Abstract
	\end{center}
	\noindent\textit{#1}
}

% Bibliografia
\newcommand{\bibliografia}[1] {
	\begingroup
	\renewcommand{\section}[2]{}%
	\begin{thebibliography}{9}
	#1
	\end{thebibliography}
	\endgroup
}